\section{Лекция 10.}
\subsection{Линейные однородные системы с постоянными
коэффициентами}
\deff{def:} \textit{Линейной системой дифференциальных уравнений с
постоянными коэффициентами} называют линейную систему вида
$$r' = Ar + q(t),$$
где $A \in \text{Mat}_n(\mathbb{C})$, $q \in C((a,b) \to \mathbb{C}^n)$.

\thmm{Лемма}

Пусть $n, k \in \mathbb{N}$, $A \in \text{Mat}_n(\mathbb{C})$, $h^1, h^2, \dots, h^k$ - жорданова цепочка, соответствующая $\lambda \in \text{spec }A$. Тогда функции
\begin{align*}
    \varphi^1(t) &= e^{\lambda t} h^1, \\
    \varphi^2(t) &= e^{\lambda t} \left( \frac{t}{1!} h^1 + h^2 \right), \\
    & \dots \\
    \varphi^k(t) &= e^{\lambda t} \left( \frac{t^{k-1}}{(k-1)!} h^1 + \dots + \frac{t}{1!} h^{k-1} + h^k \right)
\end{align*}
являются решениями системы $r' = Ar$.

\textbf{Доказательство.}

Принимая во внимание определение жордановой цепочки, при $j \in [1:k]$ имеем
\begin{align*}
    A \varphi^j &= e^{\lambda t} \sum_{m=1}^{j} \frac{t^{j-m}}{(j-m)!} A h^m = e^{\lambda t} \left( \frac{t^{j-1}}{(j-1)!} \lambda h^1 + \sum_{m=2}^{j} \frac{t^{j-m}}{(j-m)!} (\lambda h^m + h^{m-1}) \right) = \\
    &= e^{\lambda t} \left( \lambda \sum_{m=1}^{j} \frac{t^{j-m}}{(j-m)!} h^m + \sum_{m=2}^{j} \frac{t^{j-m}}{(j-m)!} h^{m-1} \right).
\end{align*}
Это же выражение получается при дифференцировании вектор-функции $\varphi^j$. Значит, $(\varphi^j)' = A \varphi^j$, что и требовалось.

\thmm{Теорема(ФСР ЛОС с постоянными коэффициентами)}

Пусть $A \in \text{Mat}_n(\mathbb{C})$, базис пространства $\mathbb{C}^n$ состоит из жордановых цепочек
\begin{align*}
    \lambda_1 &\sim h^1, h^2, \dots, h^{k}, \\
    & \dots \\
    \lambda_d &\sim u^1, u^2, \dots, u^{m},
\end{align*}
соответствующих $\lambda_1, \dots, \lambda_d \in \text{spec }A$. Тогда вектор-функции
\begin{align*}
\varphi^1(t) &= e^{\lambda_1 t} h^1, && \dots, && \varphi^k(t) = e^{\lambda_1 t} \left( \frac{t^{k-1}}{(k-1)!} h^1 + \dots + \frac{t}{1!} h^{k-1} + h^k \right), \\
    & && \dots \\
    \psi^1(t) &= e^{\lambda_d t} u^1, && \dots, && \psi^m(t) = e^{\lambda_d t} \left( \frac{t^{m-1}}{(m-1)!} u^1 + \dots + \frac{t}{1!} u^{m-1} + u^m \right)
\end{align*}
образуют фундаментальную систему решений системы $r' = Ar$.

\textbf{Доказательство.}

По лемме каждая из вектор-функций
$$\varphi^1, \dots, \varphi^k, \psi^1, \dots, \psi^m$$
является решением. Их вронскиан
$$W(0) = \det[h^1, \dots, h^k, u^1, \dots, u^m] \neq 0.$$
Тогда по теореме о критерии линейной независимости решений ЛОС вектор-функции $\{\varphi^1, \dots, \varphi^k, \psi^1, \dots, \psi^m\}$ линейно независимы, а значит, образуют фундаментальную систему решений.

\textbf{Пример(случай собственного базиса).}

Найдём общее решение системы
$$\begin{cases}
    x' = y+z, \\
    y' = x-z, \\
    z' = x-y.
\end{cases}$$
Матрица системы
$$A = \begin{bmatrix}
    0 & 1 & 1 \\
    1 & 0 & -1 \\
    1 & -1 & 0
\end{bmatrix}.$$
Её собственные числа: $-2$ (кратности $1$) и $1$ (кратности $2$). Соответствующие им собственные векторы: $[-1,1,1]^T, [1,1,0]^T, [1,0,1]^T$. Тогда общее решение
$$r(t) = C_1 e^{-2t} \begin{bmatrix} -1 \\ 1 \\ 1 \end{bmatrix} + C_2 e^{t} \begin{bmatrix} 1 \\ 1 \\ 0 \end{bmatrix} + C_3 e^{t} \begin{bmatrix} 1 \\ 0 \\ 1 \end{bmatrix}.$$

\textbf{Следствие}

Пусть $\lambda \in \text{spec }A$ имеет алгебраическую кратность $m_a$ и геометрическую кратность $m_g$. Тогда система $r' = Ar$ имеет $m_a$ линейно независимых решений вида
$$\varphi(t) = e^{\lambda t} Q^{m_a-m_g}(t),$$
где $Q^s$ --- вектор-многочлен степени не выше $s$.

\textbf{Доказательство.}

% умничка, спасибо, что помогаешь <3 люю

По теореме о ФСР ЛОС с постоянными коэффициентами числу $\lambda$ соответствуют $m_g$ групп решений размеров $k_1, k_2, \dots, k_{m_g}$, причём $k_1 + k_2 + \dots + k_{m_g} = m_a$. Все эти решения линейно независимы и имеют вид экспоненты, умноженной на некоторый вектор-многочлен. При этом в $j$-й группе степень многочленов, умножаемых на $e^{\lambda t}$, не превосходит $k_j - 1$.

Не умаляя общности, считаем $k_1 = \max_{j \in [1:m_g]} k_j$. Тогда степень многочленов не превосходит $k_1 - 1$. Так как

$$m_a = k_1 + \dots + k_{m_g} \ge k_1 + (m_g - 1),$$
то $k_1 - 1 \le m_a - m_g$, что и требовалось.

На этом следствии основан \textit{метод Эйлера} построения общего решения линейного однородного уравнения. Каждому собственному числу сопоставляется вектор-функция с неопределёнными коэффициентами. Они определяются подстановкой функции $\varphi$ в систему уравнений. Среди коэффициентов всегда будет $m$ независимых, где $m$ --- алгебраическая кратность собственного числа.

Если алгебраическая и геометрическая кратности совпадают, то метод Эйлера фактически сводится к поиску собственных векторов. Продемонстрируем данный метод на примере.

\textbf{Пример} 

Решим систему
$$\begin{cases}
    x' = 2x + y + z, \\
    y' = -z - 2x, \\
    z' = y + 2x + 2z.
\end{cases}$$
Матрица коэффициентов
$$A = 
\begin{bmatrix}
    2 & 1 & 1 \\
    -2 & 0 & -1 \\
    2 & 1 & 2
\end{bmatrix}.$$
имеет собственные числа $\lambda_1 = 2$, $\lambda_{2,3} = 1$.

Так как алгебраическая и геометрическая кратности числа $\lambda_1$ совпадают и равны $1$, то найдём собственный вектор, отвечающий $\lambda_1$: $h^1 = [1,-2,2]^T$. Тогда соответствующие решения

$$\varphi^1(t) = C_1 
\begin{bmatrix}
1 \\ -2 \\ 2 
\end{bmatrix} e^{2t}.$$
Геометрическая кратность числа $\lambda_{2,3}$ равна
$$m_g = n - \text{rank}(A - \lambda_{2,3}E) = 3 - 2 = 1.$$

Алгебраическая кратность $m_a = 2$. Поэтому многочлены в формуле имеют степень не выше $m_a - m_g = 2 - 1 = 1$. Следовательно, решения, отвечающие $\lambda_{2,3}$, ищем в виде

$$\varphi^{2,3}(t) = e^t(at+b),$$
где $a = [a_1, a_2, a_3]^T$, $b = [b_1, b_2, b_3]^T$. Подставляя $\varphi^{2,3}$ в систему, находим
$$e^t(at+b) + e^t a = e^t(tAa + Ab) \iff at + (a+b) = tAa + Ab.$$
Приравнивая коэффициенты при одинаковых степенях $t$, получаем систему
$$\begin{cases}
    Aa = a, \\
    Ab = a + b.
\end{cases}$$
Из первого уравнения находим $a_1 = 0$, $a_2 = -C_2$, $a_3 = C_2$.

Подставляя во второе уравнение системы, получаем $b_1 = C_2$, $b_2 = -C_2 - C_3$, $b_3 = C_3$. Здесь $C_2$ и $C_3$ --- произвольные параметры.

Тогда числу $\lambda_{2,3}$ соответствуют решения вида
$$\varphi^{2,3}(t) = e^t \left( t 
\begin{bmatrix} 
0 \\ -C_2 \\ C_2 
\end{bmatrix}
+ 
\begin{bmatrix} 
C_2 \\ -C_2 - C_3 \\ C_3 
\end{bmatrix} \right) = C_2 e^t 
\begin{bmatrix} 
1 \\ -t-1 \\ t 
\end{bmatrix} + C_3 e^t 
\begin{bmatrix} 
0 \\ -1 \\ 1 
\end{bmatrix}.$$
Общее решение исходной системы --- сумма $\varphi^1$ и $\varphi^{2,3}$:
$$r(t) = C_1 
\begin{bmatrix} 
1 \\ -2 \\ 2 
\end{bmatrix} e^{2t} + C_2 
\begin{bmatrix} 
1 \\ -t-1 \\ t 
\end{bmatrix} e^{t} + C_3 
\begin{bmatrix} 
0 \\ -1 \\ 1 
\end{bmatrix} e^{t}.$$
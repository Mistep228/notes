\section{Лекция 9.}
\subsection{Линейная система и её решение}
\deff{def:} Линейной системой дифференциальных уравнений называют систему вида
$$ r' = P(t)r + q(t). $$

\subsubsection{Теорема о существовании и единственности максимального решения ЛС.}

Пусть $P \in \text{Mat}_n(\mathbb{C}((a,b) \to \mathbb{R}))$, $q \in \mathbb{C}((a,b) \to \mathbb{R}^n)$, $t_0 \in (a,b)$, $r^0 \in \mathbb{R}^n$. Тогда максимальное решение задачи Коши
$$ r' = P(t)r + q(t), \quad r(t_0) = r^0 $$
существует, единственно и определено на интервале $(a, b)$.

\textbf{Доказательство.} 

Заметим, что правая часть системы $f(t, r) = P(t)r + q(t)$ и её производная $f'_r = P(t)$ непрерывны в области $(a, b) \times \mathbb{R}^n$. Тогда по теореме существует единственное максимальное решение задачи.
Имеем
$$ |f(t,r)| \le |P(t)r| + |q(t)| \le n|P(t)| \cdot |r| + |q(t)|. $$
Так как функции $u(t) = n|P(t)|$ и $v(t) = |q(t)|$ непрерывны на $(a, b)$, то по теореме решение задачи продолжимо на интервал $(a, b)$.
\subsubsection{Теорема о существовании и единственность максимального решения ЛС с комплексными коэффициентами}
Пусть $P \in \text{Mat}_n(\mathbb{C}(a, b))$, $q \in \mathbb{C}((a,b) \to \mathbb{C}^n)$, $t_0 \in (a, b)$, $r^0 \in \mathbb{C}^n$. Тогда максимальное решение задачи Коши
$$ r' = P(t)r + q(t), \quad r(t_0) = r^0 $$
существует, единственно и определено на интервале $(a, b)$.

\textbf{Доказательство.} 
Пусть
$$ P = A + iB, \quad q = \alpha + i\beta, \quad r = u + iv, \quad r^0 = u_0 + iv_0, $$
где $A, B \in \text{Mat}_n(\mathbb{C}((a, b) \to \mathbb{R}))$, $\alpha, \beta, u, v \in \mathbb{C}((a,b) \to \mathbb{R}^n)$, $u_0, v_0 \in \mathbb{R}^n$.

\textit{Единственность.} Пусть $r$ - максимальное решение задачи . Тогда
$$ u' + iv' = (A + iB)(u + iv) + \alpha + i\beta, \quad u(t_0) + iv(t_0) = u_0 + iv_0, $$
что равносильно
$$
\begin{bmatrix} u' \\ v' \end{bmatrix} =
\begin{bmatrix} A & -B \\ B & A \end{bmatrix}
\begin{bmatrix} u \\ v \end{bmatrix} +
\begin{bmatrix} \alpha \\ \beta \end{bmatrix}, \quad
\begin{bmatrix} u(t_0) \\ v(t_0) \end{bmatrix} =
\begin{bmatrix} u_0 \\ v_0 \end{bmatrix}.
$$
Значит, вектор $(u, v)^T$ - решение задачи с вещественными коэффициентами. По теореме  задача не может иметь более одного максимального решения. Поэтому, если решение задачи существует, то оно единственно.

\textit{Существование}. По теореме о существовании и единственности максимального решения ЛС задача имеет максимальное решение $(u, v)^T$, заданное на $(a, b)$. Поскольку соотношения равносильны, получаем, что $r = u + iv$ - решение задачи на $(a, b)$. Решение $r$ непродолжимо, иначе решение $(u, v)^T$ задачи было бы продолжимо.

\textbf{Замечание.} В дальнейшем под решением линейной системы подразумевается максимальное решение.
\subsection{Линейные однородные системы}
\deff{def:} Если $q \equiv 0$ на $(a, b)$, то система, то есть
$$ r' = P(t)r, $$
называется \textit{однородной}, в противном случае - \textit{неоднородной.}

\deff{def:} \textit{Определителем Вронского (вронскианом)} вектор-функций $(r^k)_{k=1}^n$, где $r^k: D \subset \mathbb{R} \to \mathbb{C}^n$, называют определитель
$$ W(t) := \det(r^1(t), r^2(t), \dots, r^n(t)). $$
\subsubsection{Лемма (свойства вронскиана решений ЛОС).}
Пусть $(r^k)_{k=1}^n$ - решения системы. Тогда следующие утверждения равносильны:
\begin{enumerate}
\item[(i)] $W(t_0) = 0$ в некоторой точке $t_0 \in (a, b)$;
\item[(ii)] $W \equiv 0$ на $(a, b)$;
\item[(iii)] $(r^k)_{k=1}^n$ линейно зависимы на $(a, b)$.
\end{enumerate}

\textbf{Доказательство.}

 Проведём доказательство по схеме: (ii) $\Rightarrow$ (i) $\Rightarrow$ (iii) $\Rightarrow$ (ii).
 
(ii) $\Rightarrow$ (i) Это следствие очевидно.

(i) $\Rightarrow$ (iii) Пункт (i) означает, что векторы $(r^k(t_0))_{k=1}^n$ линейно зависимы. Значит, найдётся набор чисел $(c_k)_{k=1}^n$, такой что
$$ \sum_{k=1}^n c_k r^k(t_0) = 0. $$

Положим $\varphi := c_1 r^1 + c_2 r^2 + \dots + c_n r^n$. Тогда $\varphi$ - решение системы , удовлетворяющее условию $\varphi(t_0) = 0$. Но решением этой же задачи Коши является функция, тождественно равная нулю на $(a, b)$. Следовательно, по теореме о существовании и единственность максимального решения ЛС с
комплексными коэффициентами будет $\varphi \equiv 0$ на $(a, b)$. Значит, вектор-функции $(r^k)_{k=1}^n$ линейно зависимы.

(iii) $\Rightarrow$ (ii) Линейная зависимость вектор-функций $(r^k)_{k=1}^n$ означает линейную зависимость столбцов матрицы $(r^1(t), r^2(t), \dots, r^n(t))$ при любом $t \in (a, b)$. Тогда её определитель, то есть вронскиан $W(t)$, тождественно равен нулю.

\subsubsection{Теорема о критерии линейной независимости решений ЛОС.}
Пусть $(r^k)_{k=1}^n$ - решения системы (6), $W$ - вронскиан данного набора. Тогда

\begin{itemize}
\item  набор $(r^k)_{k=1}^n$ линейно зависим, если и только если $W(t_0) = 0$ для некоторого $t_0 \in (a, b)$;
\item набор $(r^k)_{k=1}^n$ линейно независим, если и только если $W(t_0) \neq 0$ для некоторого $t_0 \in (a, b)$.
\end{itemize}
\subsubsection{Теорема (формула Остроградского–Лиувилля для решений ЛОС).}
Пусть $t, t_0 \in (a, b)$, $P \in \text{Mat}_n(\mathbb{C}(a, b))$, $r^1, r^2, \dots, r^n$ - решения системы. Тогда их вронскиан равен
$$ W(t) = W(t_0) \exp \int_{t_0}^t \text{tr } P(\tau) d\tau. $$
\textbf{Доказательство.}

Пусть $r$ - матрица со столбцами $r^1, r^2, \dots, r^n$, а $r_k$ - её $k$-я строка. Используя формулу полного разложения определителя, нетрудно убедиться, что
$$ W' = (\det r)' = \det \begin{bmatrix} r_1' \\ r_2 \\ \vdots \\ r_n \end{bmatrix} + \det \begin{bmatrix} r_1 \\ r_2' \\ \vdots \\ r_n \end{bmatrix} + \dots + \det \begin{bmatrix} r_1 \\ r_2 \\ \vdots \\ r_n' \end{bmatrix}. $$
Так как
$$ r' = [r^{1'}, r^{2'}, \dots, r^{n'}] = [Pr^1, Pr^2, \dots, Pr^n] = Pr, $$
то $k$-я строка матрицы $r'$ совпадает с $k$-й строкой матрицы $Pr$, то есть
$$ r_k' = P_k r = \sum_{j=1}^n P_k^j r_j. $$
Подставляя выражения для $r_k'$ при $k \in [1:n]$ в формулу для $W'$ и пользуясь тем, что определитель - линейная функция своих строк, находим
$$ W' = P_1^1 \det \begin{bmatrix} r_1 \\ r_2 \\ \vdots \\ r_n \end{bmatrix} + P_2^2 \det \begin{bmatrix} r_1 \\ r_2 \\ \vdots \\ r_n \end{bmatrix} + \dots + P_n^n \det \begin{bmatrix} r_1 \\ r_2 \\ \vdots \\ r_n \end{bmatrix} = W \text{ tr } P. $$
Решая полученное дифференциальное уравнение, приходим к требуемой формуле.
\subsubsection{Теорема (общее решение ЛОС).}
Пусть $P \in \text{Mat}_n(\mathbb{C}(a, b))$. Тогда множество решений системы $r' = P(t)r$ образует $n$-мерное линейное пространство.

\textbf{Доказательство.}

Пусть $t_0 \in (a, b)$, $(a^k)_{k=1}^n$ - базис в $\mathbb{C}^n$. По теореме о существовании и единственности максимального решения ЛС с комплексными коэффициентами для любого $k \in [1:n]$ существует $r^k$ - решение задачи Коши $r' = P(t)r, r(t_0) = a^k$. Вронскиан этих решений $W(t_0) = \det(a^1, a^2, \dots, a^n) \neq 0$. Тогда по теореме о критерии линейной независимости решений ЛОС функции $(r^k)_{k=1}^n$ линейно независимы.

Рассмотрим произвольное решение $r$ системы $r' = P(t)r$. Пусть $(c_k)_{k=1}^n$ - координаты вектора $r(t_0)$ в базисе $(a^k)_{k=1}^n$. Положим
$$ \varphi = c_1 r^1 + c_2 r^2 + \dots + c_n r^n. $$

Ясно, что $\varphi$ - решение системы $r' = P(t)r$, при этом $\varphi(t_0) = r(t_0)$. Тогда $r \equiv \varphi$ в силу теоремы о существовании и единственности максимального решения ЛС с комплексными коэффициентами.

Таким образом, функции $(r^k)_{k=1}^n$ линейно независимы, и любое решение есть их линейная комбинация. Значит, $(r^k)_{k=1}^n$ - базис в пространстве решений.

\deff{def:} \textit{Фундаментальной системой решений} системы уравнений называется совокупность её $n$ линейно независимых решений.

\deff{def:} \textit{Фундаментальная матрица системы} - матрица, столбцы которой образуют фундаментальную систему решений.

\subsubsection{Лемма о множестве фундаментальных матриц.}
Пусть $\Phi$ - фундаментальная матрица системы. Тогда $\{\Phi M \mid M \in \text{Mat}_n(\mathbb{C}), \det M \neq 0\}$ - множество всех фундаментальных матриц этой системы.

\textbf{Доказательство.}

Пусть $\Psi$ - фундаментальная матрица системы. Тогда каждый её столбец, будучи решением этой системы, является линейной комбинацией столбцов матрицы $\Phi$. Записывая коэффициенты разложения в столбцы матрицы $M$, имеем $\Psi = \Phi M$. А так как $\det \Psi \neq 0$ и $\det \Phi \neq 0$, то и $\det M \neq 0$.

Обратно, пусть $M \in \text{Mat}_n(\mathbb{C})$ - произвольная невырожденная матрица. Тогда матрица $\Phi M$ состоит из решений, а её определитель не обращается в ноль. Следовательно, по теореме о критерии линейной независимости решений ЛОС эти решения линейно независимы, поэтому $\Phi M$ - фундаментальная матрица.

\subsubsection{Лемма об овеществлении.}
Пусть $n \in \mathbb{N}$, $\Phi = (r^1, r^2, r^3, \dots, r^n)$ - фундаментальная матрица системы, при этом $r^1 = \overline{r^2}$. Тогда
$$ \Psi = (\text{Re } r^1, \text{Im } r^1, r^3, \dots, r^n) $$
- фундаментальная матрица той же системы.

\textbf{Доказательство.}

Так как
$$ \text{Re } r^1 = \frac{1}{2} (r^1 + \overline{r^1}) = \frac{1}{2} r^1 + \frac{1}{2} r^2, $$
$$ \text{Im } r^1 = \frac{1}{2i} (r^1 - \overline{r^1}) = \frac{1}{2i} r^1 - \frac{1}{2i} r^2, $$
то
$$ \Psi = \Phi \begin{bmatrix} \frac{1}{2} & \frac{1}{2i} & 0 \\ \frac{1}{2} & -\frac{1}{2i} & 0 \\ 0 & 0 & E_{n-2} \end{bmatrix}, $$
где $E_{n-2}$ - единичная матрица порядка $n-2$. По лемме о множестве фундаментальных матриц матрица $\Psi$ является фундаментальной.

\textbf{Пример.}

Рассмотрим систему
$$ x' = y, \quad y' = -x. $$

В качестве её фундаментальной матрицы можно взять
$$ \Phi = \begin{bmatrix} e^{it} & e^{-it} \\ ie^{it} & -ie^{-it} \end{bmatrix}. $$

Столбцы матрицы $\Phi$ комплексно-сопряжены. По лемме об овеществлении матрица
$$ \Psi = \begin{bmatrix} \cos t & \sin t \\ -\sin t & \cos t \end{bmatrix}, $$
столбцы которой суть вещественная и мнимая части первого столбца матрицы $\Phi$, также является фундаментальной.
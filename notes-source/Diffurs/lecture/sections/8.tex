\section{Лекция 8.}
\deff{def:} Решение $\varphi$ уравнения $r' = f(t, r)$ продолжимо, если существует решение $\psi$ того же уравнения, такое что $\text{dom}\,\varphi \subseteq \text{dom}\,\psi$ и $\psi|_{\text{dom}\,\varphi} = \varphi$. Решение $\psi$ называют \textbf{продолжением решения} $\varphi$.

\deff{def:} Если для решения $\varphi$ уравнения $r' = f(t, r)$ не существует продолжения, то функция $\varphi$ — \textbf{максимальное решение} этого уравнения.

\subsection{Теорема (критерий продолжимости).}
Пусть область $G \subset \mathbb{R}_{t,r}^{n+1}$, $f \in C(G \to \mathbb{R}^n)$,  $\varphi$ — решение уравнения $r' = f(t, r)$ на промежутке $[a, b)$. Тогда решение $\varphi$ продолжимо на отрезок $[a, c]$ при некотором $c > b$, если и только если $(b, \varphi(b -)) \in G$.

\textbf{Доказательство.} 
\textit{Необходимость}. Пусть $\psi$ — продолжение на $[a, c]$ решения $\varphi$. Тогда в силу непрерывности функции $\psi$
$$\varphi(b-) = \psi(b-) = \psi(b).$$
Поскольку $b \in [a, c]$, то из определения решения следует, что $(b, \psi(b)) \in G$.

\textit{Достаточность}. По теореме Пикара существует решение $\chi$ задачи
$$ r' = f(t, r), \, r(b) = \varphi(b-) $$
на некотором отрезке $[b, b+h]$. Положим
$$\psi(t) := \begin{cases} \varphi(t), & t \in [a, b), \\ \chi(t), & t \in [b, b+h]. \end{cases}$$
По лемме гладкой стыковки функция $\psi$ — решение уравнения $r' = f(t, r)$ на $[a, b+h]$. Тогда $\psi$ — продолжение решения $\varphi$ на $[a, c]$, где $c = b+h$.

\subsection{Теорема (существование и единственность максимального решения).}
Пусть область $G \subset \mathbb{R}_{t,r}^{n+1}$, $f \in C(G \to \mathbb{R}^n) \cap {\text{Lip}_{r,loc}}$, $(t_0, r_0) \in G$. Тогда
\begin{enumerate}
    \item[(i)] существует максимальное решение $\psi$ задачи Коши
     \[ r' = f(t,r), \quad r(t_0) = r_0; \]
    \item[(ii)] любое решение задачи — сужение решения $\psi$.
\end{enumerate}

\textbf{Доказательство.}
\begin{enumerate}
    \item[(i)] Пусть $S$ — множество всевозможных решений задачи , определённых на произвольных промежутках. По теореме Пикара это множество не пусто. Обозначим через $a_\varphi$ и $b_\varphi$ левый и правый конец промежутка $\text{dom}\,\varphi$. Положим
    \[ a = \inf_{\varphi \in S} a_\varphi, \quad b = \sup_{\varphi \in S} b_\varphi. \]
    Определим на $(a, b)$ функцию $\psi$ следующим образом. Пусть $t \in [t_0, b)$. Возьмём произвольное решение $\varphi$, для которого $t < b_\varphi$ (такое решение найдётся в силу определения числа $b$). Положим $\psi(t) = \varphi(t)$.

    Если найдётся ещё одно решение $\varphi_1$, такое что $t < b_{\varphi_1}$, то $\varphi(t) = \varphi_1(t)$ по теореме Пикара. Тем самым, в точке $t$ функция $\psi$ определена однозначно. 
    
    Из определения функции $\psi$ следует, что $\psi \equiv \varphi$ на $[t_0, b_\varphi)$.
    
    Тогда $\psi$ — решение на $[t_0, b_\varphi)$ задачи. Следовательно, $\psi'(t) = f(t, \psi(t))$.
    Так как $t$ выбиралось произвольно из $[t_0, b)$, то последнее равенство верно на $[t_0, b)$.
    
    Аналогично поступаем при $t \in (a, t_0]$. По лемме о гладкой стыковке функция $\psi$ будет решением на $(a, b)$.
    
    Продолжимость решения $\psi$ вправо за точку $b$ противоречила бы определению числа $b$. Аналогично для точки $a$. Таким образом, $\psi$ — максимальное решение.

    \item[(ii)] Пусть $\varphi \in S$. По теореме Пикара будет $\psi \equiv \varphi$ на $\text{dom}\,\psi \cap \text{dom}\,\varphi = (a_\varphi, b_\varphi)$. Так как $(a_\varphi, b_\varphi) \subset (a,b), $ то $\varphi$ — сужение функции $\psi$.
\end{enumerate}

\subsection{Теорема о выходе интегральной кривой за приделы компакта.}
Пусть $n \in \mathbb{N}$, область $G \subset \mathbb{R}_{t,r}^{n+1}$, $f \in C(G \to \mathbb{R}^n) \cap {\text{Lip}_{r,loc}}$, $\varphi$ — максимальное решение на $(a,b)$ уравнения $r' = f(t,r)$, $K \subset G$ — компакт. Тогда найдётся $\Delta > 0$, такое что $\varphi(t) \notin K$ при всех $t \in (a, a+\Delta) \cup (b-\Delta, b)$.

\textbf{Доказательство.}

Заметим, что расстояние $\rho = \rho(K, \partial G)$ от компакта $K$ до границы $\partial G$ области $G$ положительно (иначе можно было бы построить последовательность точек из $K$, сходящуюся к точке на границе, но $\partial G \cap K = \emptyset$). Если $\rho < +\infty$, положим $\rho := \rho/2$, иначе пусть $c := 1$.

Вокруг каждой точки $p' \in K$ построим внутри $G$ параллелепипед
\[ \Pi(p') = \{p \in \mathbb{R}^{n+1} \mid |p - p'| \leq c \} \]
и рассмотрим множество
\[ K_c := \bigcup_{p' \in K} \Pi(p'). \]
Поскольку $K$ — компакт, то норма каждого элемента из $K$ ограничена некоторым числом $d$. Если $p$ — произвольная точка из $K_c$, то для некоторой точки $p' \in K$ будет $p \in \Pi(p')$, поэтому
\[ |(t,r)| \leq |(t,r) - (t',r')| + |(t',r')| \leq c + d. \]
Значит, множество $K_c$ ограничено. 

Докажем его замкнутость. Рассмотрим последовательность $(p_m)$ точек из $K_c$. Для каждой такой точки найдётся параллелепипед $\Pi(p'_m)$, которому она принадлежит. Раз $K$ — компакт, то существует подпоследовательность $(p'_{m_k})$, сходящаяся к некоторой точке $p' \in K$. Переходя к пределу в неравенствах
\[ |p_{m_k} - p'_{m_k}| \leq c, \]
находим $|p - p'| \leq c$. Следовательно, $p \in K_c$.

Таким образом, $K_c$ — компакт, и функция $|f|$ достигает на нём максимального значения
\[ M := \max_{p \in K_c} |f(p)|. \]
Теперь предположим, что утверждение теоремы неверно. Пусть $\Delta = h/2$, где $h = \min \{c, c/M\}$. Тогда при некотором $t_0 \in (b-h/2, b)$ будет $(t_0, \varphi(t_0)) \in K$.

Рассмотрим задачу Коши $r' = f(t,r)$ с начальными данными $r(t_0) = \varphi(t_0)$. По теореме Пикара она имеет решение $\psi$ на отрезке $[t_0, t_0+h]$. Пусть
\[ \tilde{\varphi}(t) := \begin{cases} \varphi(t), & \text{если } t \in (a, t_0), \\ \psi(t), & \text{если } t \in [t_0, t_0+h]. \end{cases} \]
По лемме $\tilde{\varphi}$ — решение уравнения $r' = f(t,r)$ на $(a, t_0+h)$. Функция $\tilde{\varphi}$ совпадает с $\varphi$ на $(a,b) \cap (a, t_0+h)$ по теореме Пикара. Но $t_0+h > b - \frac{h}{2} + h = b + \frac{h}{2} > b$, то есть $\tilde{\varphi}  \,-  $ продолжение $\varphi$ вправо за точку $b$. Так как $\varphi$ по условию является максимальным решением, приходим к противоречию.

\subsection{Теорема о системе, сравнимой с линейной.}

Пусть $G = (a, b) \times \mathbb{R}^n_r$, $f \in C(G \to \mathbb{R}^n) \cap  {\text{Lip}_{r,loc}}$, функции $u, v \in C(a, b)$ таковы, что для любой точки $(t,r) \in G$
\[ |f(t,r)| \leq u(t)|r| + v(t). \]
Тогда каждое максимальное решение уравнения $r' = f(t,r)$ определено на $(a, b)$.

\textbf{Доказательство.}

По теореме о существование и единственности максимального решения любая задача Коши с начальными данными $(t_0, r_0) \in G$ имеет единственное максимальное решение $\varphi$, заданное на некотором интервале $(\alpha, \beta)$. Пусть, например, $\beta < b$. Применяя равносильное интегральное уравнение (лемма ??), при $t \in [t_0, \beta)$ находим
$$|\varphi(t)| = \left|r_0 + \int_{t_0}^{t} f(\tau, \varphi(\tau)) d\tau\right| \leq |r_0| + \int_{t_0}^{t} |f(\tau, \varphi(\tau))| d\tau \\ \leq |r_0| + \int_{t_0}^{t} (u(\tau)|\varphi(\tau)| + v(\tau)) d\tau.$$
Из непрерывности функций $u$ и $v$ вытекает их ограниченность на отрезке $[t_0, \beta]$. Следовательно, найдутся такие числа $\lambda \geq 0, \mu \geq 0$, что при $t \in [t_0, \beta)$
$$|\varphi(t)| \leq \lambda + \mu \int_{t_0}^{t} |\varphi(s)| ds.$$
Тогда по лемме Гронуолла
$$|\varphi(t)| \leq \lambda e^{\mu(t-t_0)} \leq L,$$
где $L = \lambda e^{\mu(\beta-t_0)}$. Отсюда следует, что график решения $\varphi$ не покидает компакт 
$$K = \{(t,r) \in G \mid t \in [t_0, \beta], |r| \leq L \} \subset G$$
при $t \in [t_0, \beta)$, что противоречит теореме о выходе интегральной кривой за пределы компакта

\textbf{Следствие.} Пусть $G = (a, b) \times \mathbb{R}^n_r$, $f \in C(G \to \mathbb{R}^n) \cap \text{Lip}_{r}$. Тогда каждое максимальное решение уравнения $r' = f(t,r)$ определено на $(a, b)$.

\textbf{Доказательство.}

Поскольку $f \in \text{Lip}_{r}$, $G$, то найдётся такое число $L > 0$, что для любых пар точек $(t,r), (t,\tilde{r}) \in G$ верно
\[ |f(t,r) - f(t, \tilde{r})| \leq L |r - \tilde{r}|. \]
Для произвольной точки $(t,r) \in G$ имеем
\[ |f(t,r)| \leq |f(t,r) - f(t, 0)| + |f(t, 0)| \leq L |r| + |f(t, 0)|. \]
Полагая $u(t) = L, v(t) = |f(t, 0)|$ в условии теоремы О системе, сравнимой с линейной, получаем требуемое.
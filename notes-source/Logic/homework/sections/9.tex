
\begin{enumerate}

\item Покажите, что омега-непротиворечивая теория непротиворечива.
\item Пусть $\zeta_\varphi(x) := \forall z.\sigma (x,x,z) \rightarrow \varphi(z)$,
где формула $\sigma(p,q,r)$ представляет функцию $\textsc{SUBST}(p,q)$, заменяющую в формуле с гёделевым номером $p$
все свободные переменные $x_1$ на формулу $q$. Тогда покажите, что формулу $\alpha_\varphi := \zeta_\varphi(\overline{\ulcorner\zeta_\varphi\urcorner})$
можно взять в качестве формулы $\alpha$ в лемме об автоссылках: $\vdash \varphi(\overline{\ulcorner\alpha_\varphi\urcorner}) \leftrightarrow \alpha_\varphi$.

\item Покажите, что если в некоторой корректной теории $\mathcal{S}$, имеющей модель $M$, ввести дополнительную аксиому $\alpha$, 
причём $\llbracket \alpha \rrbracket_M = \text{И}$, то тогда получившаяся теория не станет противоречивой и 
будет иметь ту же модель $M$ и те же оценки для формул, что и исходная.

\item Покажите, что вопрос о принадлежности формулы $\alpha(x) = \forall p.\delta(x,p) \rightarrow \neg \sigma(p)$ в доказательстве 
теоремы о невыразимости доказуемости к множеству $Th_\mathcal{S}$ ведёт к противоречию.
\item Покажите, что формула $D(x)$ из доказательства теоремы о невыразимости доказуемости является представимой в формальной арифметике.

\item Рассмотрим определение предела последовательности: $$\forall \varepsilon > 0.\exists N \in \mathbb{N}.\forall n\in\mathbb{N}. n > N \rightarrow |a_n-l| < \varepsilon$$
Раскройте все нелогические предикатные и функциональные символы, переведите эту формулу на язык исчисления предикатов, 
постройте эквивалентную формулу с поверхностными кванторами, проведите её сколемизацию
и постройте эквивалентную систему дизъюнктов.

\item Рассмотрим формулы $\forall n.P(n)\rightarrow Q(n)$ и $\forall n.P(n)\rightarrow P(f(n)) \vee P(g(n))$, здесь $P$ и $Q$ --- некоторые предикатные
символы. Постройте для каждой из них эрбранов универсум и система основных примеров.

\item Принципом Дирихле (<<pigeonhole principle>>) называется утверждение о том, что нельзя разместить $n$ кроликов в $m$ ящиках (при $m < n$) так, чтобы
каждый кролик находился бы в ящике один. 

Пусть пропозициональные переменные $P_{i,j}$, где $i \in \overline{1,n}$ и $j \in \overline{1,m}$ 
соответствуют утверждениям вида <<кролик $i$ находится в ящике $j$>>.
Формализуйте в исчислении высказываний условие 
<<каждый кролик находится в отдельном ящике в одиночестве>>,
понимаемое как условие на переменные $P_{i,j}$,
постройте соответствующее выражение в КНФ. 

Какова будет его система основных примеров?
Покажите, что система основных примеров формулы противоречива при $m < n$.
\end{enumerate}
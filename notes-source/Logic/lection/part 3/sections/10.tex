
\section{Лекция 10.}
\subsection{Введение в теорию множеств}

\deff{def:} \textbf{Теория множеств} --- теория первого порядка, с дополнительным нелогическим двухместным функциональным символом $\in$, и следующими дополнительными нелогическими аксиомами и схемами аксиом.


\subsection{Аксиоматика ZF, равенство,  конструктивные аксиомы}

\deff{def:} \textbf{Равенство <<по Лейбницу>>}: объекты равны, если неразличимы. Если нечто ходит как утка, выглядит как утка и крякает как утка, то это утка.

\deff{def:} \textbf{Принцип объёмности}: объекты равны, если состоят из одинаковых частей.

$A \subseteq B \equiv \forall x.x \in A \rightarrow x \in B$ 
 $A = B \equiv A \subseteq B \with B \subseteq A$ 

\deff{def:} \textbf{Аксиома равенства}: равные множества содержатся в одних и тех же множествах. $\forall x. \forall y. \forall z. x = y \with x \in z \rightarrow y \in z$.

\deff{def:} \textbf{Аксиома пустого.} Существует пустое множество $\varnothing$. $$\exists s.\forall t.\neg t \in s$$ 
\deff{def:} \textbf{Аксиома пары.} Существует $\{a,b\}$. Каковы бы ни были два множества $a$ и $b$, существует множество, состоящее в точности из них. 
$$\forall a.\forall b.\exists s.a \in s \with b \in s \with \forall c.c \in s \rightarrow c = a \vee c = b$$ 
\deff{def:} \textbf{Аксиома объединения:} существует $\cup x$. Для любого непустого множества $x$ найдется такое множество, состоящее в точности из тех элементов, из которых состоят элементы $x$. 
$$\forall x.(\exists y.y \in x) \rightarrow \exists p.\forall y.y \in p \leftrightarrow \exists s.y \in s \with s \in x$$
\deff{def:} \textbf{Аксиома степени}: существует $\mathcal{P}(x)$. Каково бы ни было множество $x$, существует множество, содержащее в точности все возможные подмножества множества $x$.
$$\forall x.\exists p.\forall y.y \in p \leftrightarrow y \subseteq x$$
\deff{def:} Схема аксиом выделения: существует $\{ t \in x\ |\ \varphi(t)\}$. Для любого множества $x$ и любой формулы от одного аргумента $\varphi(y)$ ($b$ не входит свободно в $\varphi$), найдется $b$, в которое входят те и только те элементы из множества $x$, что $\varphi(y)$ истинно.
$$\forall x.\exists b.\forall y.y \in b \leftrightarrow (y \in x \with \varphi(y))$$



\thmm{Теорема.}

Для любого множества $X$ существует множество $\{X\}$, содержащее в точности $X$.

Воспользуемся аксиомой пары: $\{X,X\}$

\thmm{Теорема}

Пустое множество единственно.

\textbf{Доказательство:}

Пусть $\forall p.\neg p \in s$ и $\forall p.\neg p \in t$. Тогда $s \subseteq t$ и $t \subseteq s$.

\hfill Q.E.D.

\thmm{Теорема.} 

Для двух множеств $s$ и $t$ существует множество, являющееся их пересечением.

\textbf{Доказательство:}

$s \cap t = \{ x\in s\ |\ x \in t\}$

\hfill Q.E.D.

\deff{def:}\textbf{Упорядоченная пара.} Упорядоченной парой двух множеств $a$ и $b$ назовём $\{\{a\},\{a,b\}\}$, или $\langle{}a,b\rangle$


\thmm{Теорема.} Упорядоченную пару можно построить для любых множеств.


\textbf{Доказательство:}

Применить аксиому пары, теорему о существовании $\{X\}$, аксиому пары.

\hfill Q.E.D.

\thmm{Теорема.} $\langle{}a,b\rangle = \langle{}c,d\rangle$ тогда и только тогда, когда $a = c$ и $b = d$.


\subsection{Аксиома бесконечности}

\deff{def:} Инкремент: $x' \equiv x \cup \{x\}$

\deff{def:} Аксиома бесконечности. Существует $N: \varnothing \in N \with \forall x.x \in N\rightarrow x' \in N$

В $N$ есть всевозможные множества вида $\varnothing$, $\{\varnothing\}$, $\{\varnothing,\{\varnothing\}\}$, $\{\varnothing,\{\varnothing\},\{\varnothing,\{\varnothing\}\}\}$, \dots

(неформально) $\omega = \{\varnothing, \varnothing', \varnothing'', \dots\}$. Тогда $N_1 = \omega\cup\{\omega,\omega',\omega'',\dots\}$ подходит.

\begin{enumerate}
\item Частичный: рефлексивность ($a \preceq a$), антисимметричность ($a \preceq b \rightarrow b \preceq a\rightarrow a=b$), транзитивность ($a \preceq b \rightarrow b \preceq c \rightarrow a \preceq c$).
\item Линейный: частичный + $\forall a.\forall b.a \preceq b \vee b \preceq a$.
\item Полный: линейный + в любом непустом подмножестве есть наименьший элемент.
\end{enumerate}

$\mathbb{Z}$ не вполне упорядочено: в $\mathbb{Z}$ нет наименьшего.

Отрезок $[0,1]$ не вполне упорядочен: $(0,1)$ не имеет наименьшего.

$\mathbb{N}$ вполне упорядочено.

\subsection{Ординалы (порядковые числа)}

\deff{def:} Транзитивное множество $X$: $\forall x.\forall y.x \in y \with y \in X \rightarrow x \in X$.

\deff{def:} Ординал (порядковое число) --- вполне упорядоченное отношением $(\in)$ транзитивное множество.

Ординалы: $\varnothing$, $\varnothing'$, $\varnothing''$, \dots

\deff{def:} Предельный ординал: такой $x$, что $x \ne \varnothing$ и нет $y: y' = x$

\deff{def:} Ординал $x$ конечный, если он сам не предельный и нет предельного, меньшего его.

\thmm{Теорема.}

Если $x,y$ --- ординалы, то $x = y$, или $x\in y$, или $y \in x$.


\deff{def:} $\omega$ --- наименьший предельный ординал.

\thmm{Теорема.} 

$\omega$ существует.

\textbf{Доказательство:}

Пусть $\omega = \{ x \in N\ |\ x\text{ конечен}\}$. Тогда:
\begin{itemize}
\item меньше $\omega$ предельных нет: если $\theta$ таков, что $\theta \in \omega$, тогда $\theta$ конечен.
\item $\omega$ предельный: Пусть $\theta$ таков, что $\theta' = \omega$. Тогда $\theta$ конечен и $\theta'$ тоже конечен.
\end{itemize}

\hfill Q.E.D.

$\omega'$ --- тоже ординал.

\deff{def:} Порядковый тип множества --- некоторое свойство, общее для всех множеств, изоморфных относительно биективных отображений, сохраняющих порядок.

\deff{def:} Порядковый тип вполне упорядоченного множества $\langle S, (\preceq)\rangle$ --- ординал $A$, для которого есть биективное отображение $f: S \rightarrow A$, сохраняющее порядок: $a \preceq b$ тогда и только тогда, когда $f(a) \le f(b)$

Множество $\mathbb{Z}$ не имеет порядкового типа (в смысле определения через ординалы): оно не вполне упорядочено.

\subsection{Операции над ординалами}

\deff{def:} $a + b$ --- порядковый тип $a \uplus b$ (отмеченного объединения), причём $x_a < y_b$ при любых $x \in a$ и $y \in b$\

\deff{def:} $a \cdot b$ --- порядковый тип $a \times b$, произведение упорядочено лексикографически: $\langle x_1, y_1 \rangle < \langle x_2, y_2 \rangle$, если $x_1 < x_2$ или $x_1 = x_2$ и $y_1 < y_2$.

$\overline{3} + \overline{4}$: порядковый тип множества $\{\overline{0}_a, \overline{1}_a, \overline{2}_a, \overline{0}_b, \overline{1}_b, \overline{2}_b, \overline{3}_b\}$, то есть $\overline{7}$

$\overline{\omega} \cdot \overline{\omega}$: порядковый тип всех натуральных точек плоскости, $\{\langle 0,0 \rangle, \dots, \langle 0,100\rangle, \dots, \langle 100,0\rangle, \dots\}$

\subsection{Операции над ординалами --- как вычислять}

\deff{def:} $\text{upb } x$ --- верхняя грань множества ординалов, $\text{upb }x = \bigcup_{a \in x} a$.

$\text{upb } \{ \varnothing', \varnothing'', \varnothing'''' \} = \varnothing' \cup \varnothing'' \cup \varnothing'''' = \{ \varnothing \} \cup \{ \varnothing, \{ \varnothing \} \} \cup \{ \varnothing, \{ \varnothing \}, \{ \varnothing, \{ \varnothing \} \}, \{ \varnothing, \{ \varnothing \}, \{ \varnothing, \{ \varnothing \} \} \} \} = \varnothing''''$

\thmm{Теорема.}
$$a + b \equiv \left\{ \begin{array}{rl} 
   a, & b \equiv \varnothing\\
   (a + c)', & b \equiv c'\\
   \text{upb } \{ a+c \mid c \prec b \}, &\mbox{$b$ --- предельный ординал }\end{array}\right.$$

$\omega + 1 = \omega \cup \{\omega\}$; $1 + \omega = \text{upb }\{ 1+\varnothing, 1+1, 1+2, \dots \} = \omega$


\thmm{Теорема:}
$$a \cdot b \equiv \left\{ \begin{array}{rl} 
   0, & b \equiv \varnothing\\
   (a \cdot c) + a, & b \equiv c'\\
   \text{upb } \{ a \cdot c \mid c \prec b \}, &\mbox{b --- предельный ординал }\end{array}\right.$$


\deff{def:}
$$a ^ b \equiv \left\{ \begin{array}{rl} 
   1, & b \equiv \varnothing\\
   (a ^ c) \cdot a, & b \equiv c'\\
   \text{upb } \{ a^c \mid c \prec b \}, &\mbox{$b$ --- предельный ординал }\end{array}\right.$$


$\omega \cdot \omega = \text{upb }\{\omega \cdot 0, \omega \cdot 1,\omega\cdot 2, \omega\cdot 3, \dots\} = \text{upb }\{0, \omega,\omega\cdot 2, \omega\cdot 3, \dots\}$

\subsection{Ординалы (порядковые числа) и порядок}

\begin{itemize}
\item Добавить элемент перед бесконечностью: $\mathbb{N}$ и $\mathbb{N}_0$.
$1 + \omega = \omega$. 
\item Добавить элемент после бесконечности $(+\infty)$. $\omega + 1 \ne \omega$ 
\end{itemize}

Упорядоченные пары натуральных чисел имеют порядковый тип $\omega^2$.

\begin{center}$\langle 3,5 \rangle < \langle 4,3 \rangle\quad\quad\omega \cdot 3 + 5 < \omega \cdot 4 + 3$.\end{center}

Списки натуральных чисел --- порядковый тип $\omega^\omega$.
$$\langle 3,1,4,1,5,9\rangle\quad\quad \omega^5 \cdot 3 + \omega^4 \cdot 1 + \omega^3 \cdot 4 + \omega^2 \cdot 1 + \omega^1 \cdot 5 + 9$$


\subsection{Дизъюнктные множества}

\deff{def:} Дизъюнктное (разделённое) множество --- множество, элементы которого не пересекаются. 
$$Dj(x) \equiv \forall y.\forall z.(y \in x \with z \in x \with \neg y=z) \rightarrow \neg \exists t.t \in y \with t \in z$$

Дизъюнктное: $\{\{1,2\},\{\rightarrow\},\{\alpha,\beta,\gamma\}\}$\\ Не дизъюнктное: $\{\{1,2\},\{\rightarrow\},\{\alpha,\beta,\gamma,1\}\}$


\deff{def:} Прямое произведение дизъюнктного множества $a$ --- множество $\times a$ всех таких множеств $b$, что:
\begin{itemize}
\item $b$ пересекается с каждым из элементов множества $a$ в точности в одном элементе
\item $b$ содержит элементы только из $\cup a$.
\end{itemize}
$$\forall b .b \in \times a \leftrightarrow (b \subseteq \cup a \with \forall y .y \in a \rightarrow \exists ! x .x \in y \with x \in b)$$

$\times\{\{\triangle,\square\},\{1,2,3\}\} = \{\{\triangle,1\},\{\triangle,2\},\{\triangle,3\},\{\square,1\},\{\square,2\},\{\square,3\}\}$

\subsection{Аксиома выбора}

\deff{def:} Прямое произведение непустого дизъюнктного множества, не содержащего пустых элементов, непусто.
$$\forall t.Dj (t) \rightarrow (\forall x.x \in t \rightarrow \exists p.p \in x) \rightarrow (\exists p.p \in \times t)$$

Альтернативные варианты: любое множество можно вполне упорядочить, любая сюръективная функция имеет частичную обратную, и т.п.

\deff{def:} Аксиоматика ZF + аксиома выбора = ZFC

\subsection{Аксиома фундирования}

\deff{def:} Аксиома фундирования. В каждом непустом множестве найдется элемент, не пересекающийся с исходным множеством.
$$\forall x .x = \varnothing \vee \exists y .y \in x \with \forall z.z \in x \rightarrow z \notin y$$


Иными словами, в каждом множестве есть элемент, минимальный по отношению $(\in)$.

Идея Рассела: каждому множеству припишем \emph{тип} (тип пустого 0, тип множеств 1, тип множеств множеств 2 и т.п.). Тогда конструкция невозможна: $\{ x\ |\ x \in x\}$. Аксиома фундирования позволяет определить функцию ранга: $$rk(x) = \text{upb }\{rk(y)\ |\ y\in x\}$$.


\deff{def:} Схема аксиом подстановки. Пусть задана некоторая функция f, представимая в исчислении предикатов: то есть задана некоторая формула $\phi$, такая, что $f(x) = y$ тогда и только тогда, когда $\phi(x,y) \with \exists ! z. \phi(x,z)$. Тогда для любого множества S существует множество f(S) --- образ множества S при отображении f.
$$\forall s .(\forall x .\forall y_1 .\forall y_2 .x \in s \with \phi (x,y_1) \with \phi (x,y_2) \rightarrow y_1=y_2) \rightarrow (\exists t .\forall y .y \in t \leftrightarrow \exists x . x \in s \with \phi (x,y)) $$


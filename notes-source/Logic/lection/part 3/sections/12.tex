
\section{Лекция 12.}
\subsection{Аксиома выбора}

\deff{def:} Аксиома выбора:

Из любого семейства дизъюнктных непустых множеств $\mathcal{A}$ можно выбрать непустую трансверсаль --- 
множество $S$, что $|S \cap A| = 1$ для каждого $A\in\mathcal{A}$. Иначе, $S \in \times \mathcal{A}$.

\thmm{Теорема: функциональный вариант аксиомы выбора}
Пусть $\mathcal{A}$ --- семейство непустых множеств. Тогда существует
$f : \mathcal{A} \rightarrow \cup \mathcal{A}$, причём $\forall a.a \in \mathcal{A} \rightarrow f(a) \in a$


\textbf{Доказательство:}

Пусть $X(A) = \{ \langle A, a \rangle \ |\ a \in A \}$, 
по семейству $\mathcal{A}$ рассмотрим $\{X(A)\ |\ A\in\mathcal{A}\}$
\begin{itemize}
\item непустых: если $A\in\mathcal{A}$, $A \ne \varnothing$, то $X(A) \ne \varnothing$;
\item дизъюнктное: если $A_0,A_1\in\mathcal{A}$, $A_0 \ne A_1$, то $X(A_0) \cap X(A_1) = \varnothing$
\end{itemize}
тогда по аксиоме выбора $\exists f.f \in \times \mathcal{A}$.

\hfill Q.E.D.

Обратное утверждение также легко показать.


\thmm{Теорема. Лемма Цорна}

Если задано $\langle M, (\preceq) \rangle$ и для всякого линейно упорядоченного $S \subseteq M$ выполнено
$\text{upb}_M S \ne \varnothing$, то в $M$ существует максимальный элемент.

\thmm{Теорема Цермело}

На любом множестве можно задать полный порядок.

\thmm{Теорема:}

У любой сюръективной функции существует частичная обратная.

\thmm{Теорема}

Аксиома выбора $\Rightarrow$ лемма Цорна: без доказательства.


\subsection{Начальный отрезок}

\deff{def:}

Назовём (для данного раздела) упорядоченным множеством пару $\langle S, (\prec_S)\rangle$.
Отношение порядка $(\prec_S)$ может быть как строгим, так и нестрогим.
Будем говорить, что $\langle S, (\prec_S)\rangle$ --- начальный отрезок $\langle T, (\prec_T) \rangle$,
если:\begin{itemize}
\item $S \subseteq T$;
\item если $a,b \in S$, то $a \prec_S b$ тогда и только тогда, когда $a \prec_T b$;
\item если $a \in S$, $b \in T\setminus S$, то $a \prec_T b$.
\end{itemize}
Будем обозначать это как $\langle S, (\prec_S)\rangle\sqsubseteq\langle T, (\prec_T)\rangle$ или как $S \sqsubseteq T$, если порядок на множествах понятен из контекста.


\thmm{Теорема.}

Отношение <<быть начальным отрезком>> является отношением нестрогого порядка.


\thmm{Теорема о верхней грани}
Если семейство упорядоченных множеств $X$ линейно упорядочено отношением <<быть начальным отрезком>>, то у него есть верхняя грань.

\textbf{Доказательство:}

Пусть $M = \cup \{ T | \langle T, (\prec) \rangle \in X \}$ и
$(\prec)_M = \cup \{ (\prec) | \langle T, (\prec) \rangle \in X \}$.

Покажем, что если $\langle A, (\prec_A)\rangle \in X$, то $A \sqsubseteq M$. Рассмотрим определение:
\begin{itemize}
\item $A \subseteq M$ --- выполнено по построению $M$;
\item если $a,b \in A$, то $a \prec_A b$ влечёт $a \prec_M b$ (по построению $M$). Если же $a \prec_M b$, но $a \not\prec_A b$,
то существует $A'$, что $a,b \in A'$ и $a \prec_{A'} b$. Тогда $A\not\sqsubseteq A'$ и $A'\not\sqsubseteq A$, что невозможно
по линейности порядка;
\item если $a \in A$, $b \in M\setminus A$, то найдётся $B$, что $b\in B$, отчего $a \prec_B b$ (так как $A \sqsubseteq B$) 
и $a \prec_M b$ (по построению $M$).
\end{itemize}

Тогда $\langle M, (\prec_M)\rangle$ --- требуемая верхняя грань.

\hfill Q.E.D.

\thmm{Теорема.}

Лемма Цорна $\Rightarrow$ теорема Цермело

Пусть выполнена лемма Цорна и дано некоторое $X$. Покажем, что на нём можно ввести полный порядок.
\begin{itemize}
\item Пусть $S = \{ \langle P, (\prec)\rangle \ |\ P \subseteq X, (\prec)\text{ --- полный порядок} \}$.
{\color{gray}Например, для $X = \{0,1\}$ множество
$S = \{
\langle\varnothing,\varnothing\rangle,
\langle \{0\},\varnothing\rangle,
\langle\{1\},\varnothing\rangle,
\langle X, 0 \prec 1\rangle,
\langle X, 1 \prec 0\rangle
\}$}

\item Введём порядок на $S$ как $(\sqsubseteq)$. Заметим, что это --- частичный, но не линейный порядок. 
{\color{gray}Например, $\langle X, 0 \prec 1\rangle$ несравним с $\langle X, 1 \prec 0\rangle$.}

\item По теореме о верхней грани любое линейно упорядоченное подмножество 
$\langle T, (\sqsubseteq) \rangle$ (где $T \subseteq S$) имеет
верхнюю грань.

{\color{gray}Например, 
для $\{\langle\varnothing,\varnothing\rangle,
\langle \{0\},\varnothing\rangle,
\langle X, 0 \prec 1\rangle\}$ это $\langle X, 0 \prec 1\rangle$.}

\item По лемме Цорна тогда есть $\langle R, (\sqsubseteq_R)\rangle = \max S$. Заметим, что $R = X$, потому что иначе пусть
$a \in X\setminus R$. Тогда положив $M = \langle R\cup\{a\}, (\sqsubseteq_R)\cup\{x\prec a\ |\ x \in R\} \rangle$
получим, что $M$ тоже вполне упорядоченное (и потому $M \in S$), значит, $R$ не максимальное.
\end{itemize}

\thmm{Теорема Цермело $\Rightarrow$ существование обратной $\Rightarrow$ аксиома выбора}

\thmm{Теорема Цермело $\Rightarrow$ у сюръективных функций существует частичная обратная.}

\textbf{Доказательство:}

Рассмотрим сюръективную $f: A \rightarrow B$. Рассмотрим семейство $R_b = \{ a \in A\ |\ f(a) = b \}$.
Построим полный порядок на каждом из $R_b$. Тогда $f^{-1}(b) = \min R_b$.


\thmm{Теорема.}

Существует частичная обратная у сюръективных функций $\Rightarrow$ существует трансверсаль у семейства непустых дизъюнктных множеств.

\textbf{Доказательство:}

Пусть дано семейство непустых дизъюнктных множеств $\mathcal{A}$. 
Рассмотрим $f: \cup \mathcal{A} \rightarrow \mathcal{A}$, что
$f(a) = \cup\{ A \in \mathcal{A}\ |\ a \in A \}$. Поскольку элементы $\mathcal{A}$ дизъюнктны,
$f(a) \in \mathcal{A}$ при всех $a$. Тогда существует $f^{-1}: \mathcal{A} \rightarrow \cup\mathcal{A}$. Тогда 
$\{ f^{-1}(A)\ |\ A\in\mathcal{A} \} \in \times \mathcal{A}$.

\hfill Q.E.D


\subsection{Равенство и функции}

Пусть $A_0 = \{0,1,3,5\}$ и $A_1 = \{3,5,1,0,0,5,3\}$.
Верно ли, что $A_0 = A_1$?

Да, так как $\forall x.x \in \{0,1,3,5\} \leftrightarrow x \in \{3,5,1,0,0,5,3\}$.

\thmm{Конгруэнтность}

Если $f: A \rightarrow B$, также $a,b\in A$ и $a=b$, то $f(a) = f(b)$.

\textbf{Доказательство:}

Пусть $F \subseteq A\times B$ --- график функции $f$.

По определению функции, $\forall x.\forall y_1.\forall y_2.\langle x,y_1\rangle \in F \with \langle x,y_2 \rangle \in F \rightarrow y_1 = y_2$.\\
Также, если $f(a) = y_1$, $f(b) = y_2$, то $\langle a,y_1 \rangle \in F$ и $\langle b,y_2 \rangle \in F$.\\
Тогда: $\langle a,y_1\rangle = \langle b,y_1\rangle = \langle b,y_2 \rangle = \langle a,y_2\rangle$,
то есть $f(a) = y_2 = f(b)$.

\hfill Q.E.D.

\subsection{Теорема Диаконеску}

Если рассмотреть ИИП с ZFC, то для любого $P$ выполнено $\vdash P \vee \neg P$.

\textbf{Доказательство:}

Рассмотрим $\mathcal{B} = \{0,1\}$, $A_0 = \{ x \in \mathcal{B} | x = 0 \vee P \}$ и 
$A_1 = \{ x \in \mathcal{B} | x = 1 \vee P\}$.
$\{A_0,A_1\}$ --- семейство непустых множеств, и по акс. выбора существует
$f: \{A_0,A_1\} \rightarrow \cup A_i$, что $f(A_i) \in A_i$. (Если $P$, то $A_0 = A_1$ и $\{A_0,A_1\} = \{\mathcal{B}\}$).

\begin{tabular}{ll}
$\vdash f(A_0) \in A_0 \with f(A_1) \in A_1$ & а.выбора: $f(A_i) \in A_i$\\
$\vdash {\color{olive}f(A_0) \in \mathcal{B}} \with (f(A_0) = 0 \vee P) \with {\color{olive}f(A_1) \in \mathcal{B}} \with (f(A_1) = 1 \vee P)$ & а.выделения\\
$\vdash (f(A_0) = 0 \with f(A_1) = 1) \vee P$ & Удал. $(\with)$ + дистр.\\
$\vdash P\vee{\color{blue}f(A_0) \ne f(A_1)}$ & $0 \ne 1$ и транз.\\
$\vdash P \rightarrow A_0 = A_1$ & Определение $A_i$\\
$\vdash A_0 = A_1 \rightarrow f(A_0) = f(A_1)$ & Конгруэнтность\\
$\vdash \color{blue} f(A_0) \ne f(A_1) \rightarrow \neg P$ & Контрапозиция\\
$\vdash P \vee \neg P$ & Подставили
\end{tabular}

\subsection{Слабые варианты аксиомы выбора}

\thmm{Теорема конечного выбора}

Если $X_1\ne\varnothing, \dots, X_n\ne\varnothing$, $X_i\cap X_j = \varnothing$ при $i \ne j$, то $\times \{X_1, \dots, X_n\} \ne \varnothing$.



\textbf{Доказательство:}

\begin{itemize}\item База: $n=1$. Тогда $\exists x_1.x_1 \in X_1$, поэтому $\exists x_1.\{x_1\} \in \times \{X_1\}$.

\item Переход: $\exists v.v \in \times \{X_{1,n}\}\rightarrow\exists x_{n+1}.x_{n+1} \in X_{n+1}\rightarrow
v \cup \{x_{n+1}\} \in \times (X_{1,n}\cup\{X_{n+1}\})$
\end{itemize}

\textbf{Аксиома счётного выбора}

Для счётного семейства непустых множеств существует функция, каждому из которых сопоставляющая один из своих элементов


\textbf{Аксиома зависимого выбора}

Если $\forall x \in E.\exists y \in E. x R y$, то существует последовательность $x_n: \forall n.x_n R x_{n+1}$


Заметим, что семейство $\{A_0, A_1\}$ из теоремы Диаконеску в ИИП не является конечным (равно как и бесконечным).
\deff{def:} Конечное множество --- равномощное некоторому конечному кардинальному числу.

\begin{itemize}
\item Какова мощность семейства? 
\item 1, если $P$, и 2, если $\neg P$. 
\item Но поскольку $P \vee \neg P$ не выполнено в ИИП, мы не можем
доказать, что мощность семейства 1 или 2.
\item Поэтому мы не можем воспользоваться теоремой конечного выбора.
\end{itemize}

\subsection{Наследственные фундированные множества}

\deff{def:} \textbf{Наследственным свойством множества} назовём такое свойство, которым обладает как само
множество, так и все его подмножества.


\deff{def:} \textbf{Фундированным множеством} назовём такое, которое не пересекается хотя бы с одним своим элементом.


\subsection{Каковы возможные модели для теории множеств?}

\deff{def:} \emph{Универсум фон Неймана} $V$ --- все наследственные фундированные множества.

При наличии аксиомы фундирования можно показать, что $V = \cup_a V_a$, где:
$$V_a = \left\{\begin{array}{ll}
    \varnothing, & a=0\\
    \mathcal{P}(V_b), & a = b'\\
    \bigcup_{b < a}(V_b), & a \text{ --- предельный}
\end{array}\right.$$

\deff{def:}
\emph{Конструктивный универсум} $L = \cup_a L_a$, где:
$$L_a = \left\{\begin{array}{ll}
    \varnothing, & a=0\\
    \{ \{ x\in L_b\ |\ \varphi(x,t_1,\dots,t_k) \}\ |\ \varphi\text{ --- формула}, t_i \in L_b\}, & a = b'\\
    \bigcup_{b < a}(L_b), & a \text{ --- пред.}
\end{array}\right.$$


\subsection{Усиление аксиомы выбора}

\deff{def:}
Аксиома конструктивности: $V=L$, то есть допустимы только те фундированные множества, которые задаются формулами.


\thmm{Теорема.} Аксиома выбора и континуум-гипотеза следуют из аксиомы конструктивности

Для некоторых теорий аксиома слишком сильна.

\subsection{Заключительный обзор}

Конструктивность теории --- насколько легко строить сложные объекты в ней:
\begin{enumerate}
\item Неконструктивные теории допускают доказательства чистого существования произвольных по сложности объектов.
\item Конструктивные теории: требуют процесс построения (желательно конечный или хотя бы счётный), 
состоящий из интуитивно понятных шагов.
\end{enumerate}

Аксиома выбора и её рассмотренные варианты влияют на её конструктивность:
\begin{enumerate}
\item КИП + ЦФ + Акс. выбора: менее конструктивна.
Например, возможно показать существование разбиения шара на 5 частей, из которых можно составить два шара,
равных исходному (теорема Банаха-Тарского).
Интуитивно нарушается аддитивность объёма (формального парадокса нет).

\item КИП + ЦФ
\item ИИП + ЦФ: более конструктивна. Она проще формализуется с помощью компьютера, но мат. анализ в ней сложнее и довольно сильно отличается от классического.
\end{enumerate}


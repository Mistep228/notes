
\section{Лекция 11}
\subsection{Отношения}

\deff{def:} $A \times B := \{\langle a,b \rangle\ |\ a \in A, b \in B\}$


Бинарное отношение --- $R \subseteq A \times B$

Функциональное бинарное отношение (функция) $R$ --- такое, что $\forall x.x\in A\rightarrow\exists !y.\langle x,y\rangle \in R$

$R$ --- инъективная функция, если $\forall x.\forall y.\langle x,t\rangle \in R\with \langle y,t\rangle \in R \rightarrow x=y$.

$R$ --- сюръективная функция, если $\forall y.y \in B\rightarrow\exists x.\langle x,y\rangle\in R$.

\subsection{Равномощные множества}

\deff{def:}Множество $A$ \emph{равномощно} $B$ $(|A|=|B|)$, если существует биекция $f: A \rightarrow B$.

Множество $A$ имеет мощность, не превышающую мощности $B$ $(|A|\le|B|)$, если существует инъекция $f: A \rightarrow B$.


\thmm{Теорема Кантора-Бернштейна}

Если $|A| \le |B|$ и $|B| \le |A|$, то $|A| = |B|$.

Заметим, $f: A \rightarrow B$, $g: B \rightarrow A$ --- инъекции, но не обязательно $g(f(x)) = x$.

\textbf{Доказательство:}

Избавимся от множества $B$: пусть $A_0 = A$; $A_1 = g(B)$; $A_{k+2} = g(f(A_k))$.

\begin{center}
\begin{tikzpicture}
\node[inner sep=0, outer sep=0] (A0) at (0,0) {};
\node[inner sep=0, outer sep=0] (A1) at (2,0) {};
\node[inner sep=0, outer sep=0] (A2) at (3,0) {};
\node[inner sep=0, outer sep=0] (A3) at (3.5,0) {};
\node[inner sep=0, outer sep=0] (A4) at (3.75,0) {};
\node[inner sep=0, outer sep=0] (AN) at (4,0) {};
\node[inner sep=0, outer sep=0] (AE) at (6,0) {};

\node (B0) at (0,-1.5) {};
\node (B1) at (2,-1.5) {};
\node (B2) at (3,-1.5) {};
\node (B3) at (3.5,-1.5) {};
\node (B4) at (3.75,-1.5) {};
\node (BN) at (4,-1.5) {};
\node (BE) at (6,-1.5) {};

\fill[gray!80] ($(A0)+(0,0.1)$) rectangle node[pos=0.1,above]{\color{gray} $A_0$} ($(AE)+(0,0.15)$);
\fill[gray!30] ($(A1)+(0,0.2)$) rectangle node[pos=0.1,above]{\color{gray} $A_1$} ($(AE)+(0,0.25)$);
\fill[gray!80] ($(A2)+(0,0.3)$) rectangle node[pos=0.1,above]{\color{gray} $A_2$} ($(AE)+(0,0.35)$);
\fill[gray!30] ($(A3)+(0,0.4)$) rectangle node[pos=0.1,above]{\color{gray} $A_3$} ($(AE)+(0,0.45)$);
\fill[gray!80] ($(AN)+(0,0.5)$) rectangle node[midway,above]{\color{gray} $\cap A_k$} ($(AE)+(0,0.55)$);

\fill[gray!80] ($(B0)-(0,0.1)$) rectangle node[pos=0.1,below]{\color{gray} $B_0$} ($(BE)-(0,0.15)$);
\fill[gray!80] ($(B1)-(0,0.1)$) rectangle node[pos=0.1,below]{\color{gray} $B_1$} ($(BE)-(0,0.15)$);
\fill[gray!80] ($(B2)-(0,0.1)$) rectangle node[pos=0.1,below]{\color{gray} $B_2$} ($(BE)-(0,0.15)$);
\fill[gray!80] ($(B3)-(0,0.1)$) rectangle node[pos=0.1,below]{\color{gray} $B_3$} ($(BE)-(0,0.15)$);
\fill[gray!80] ($(BN)-(0,0.1)$) rectangle node[midway,below]{\color{gray} $\cap B_k$} ($(BE)-(0,0.15)$);

\fill[gray!30] (4,0) -- (6,0) -- (6,-1.5) -- (4,-1.5);
\fill[gray!30] (3,-1.5) -- (3.5,-1.5) -- (3.75,0) -- (3.5,0);
\fill[gray!30] (0,-1.5) -- (2,-1.5) -- (3,0) -- (2,0);
\fill[gray!30] (3.75,-1.5) -- (3.875,-1.5) -- (3.875+0.0625,0) -- (3.875,0);
\fill[gray!80] (0,0) -- (2,-1.5) -- (3,-1.5) -- (2,0);
\fill[gray!80] (3,0) -- (3.5,-1.5) -- (3.75,-1.5) -- (3.5,0);
\fill[gray!80] (3.75,0) -- (3.875,-1.5) -- (3.875+0.0625,-1.5) -- (3.875,0);
\end{tikzpicture}
\end{center}

Тогда, если существует $h: A_0 \rightarrow A_1$ --- биекция, то тогда $g^{-1}\circ h: A \rightarrow B$ --- требуемая биекция.



Пусть $C_k = A_k \setminus A_{k+1}$. Тогда $g(f(C_k)) = g(f(A_k))\setminus g(f(A_{k+1})) = A_{k+2}\setminus A_{k+3} = C_{k+2}$.

\begin{center}
\begin{tikzpicture}
\node[inner sep=0, outer sep=0] (A0) at (0,0) {};
\node[inner sep=0, outer sep=0] (A1) at (2,0) {};
\node[inner sep=0, outer sep=0] (A2) at (3,0) {};
\node[inner sep=0, outer sep=0] (A3) at (3.5,0) {};
\node[inner sep=0, outer sep=0] (A4) at (3.75,0) {};
\node[inner sep=0, outer sep=0] (AN) at (4,0) {};
\node[inner sep=0, outer sep=0] (AE) at (6,0) {};

\node (B0) at (0,-1.5) {};
\node (B1) at (2,-1.5) {};
\node (B2) at (3,-1.5) {};
\node (B3) at (3.5,-1.5) {};
\node (B4) at (3.75,-1.5) {};
\node (BN) at (4,-1.5) {};
\node (BE) at (6,-1.5) {};

\fill[gray!80] ($(A0)+(0,0.1)$) rectangle node[pos=0.1,above]{\color{gray} $C_0$} ($(AE)+(0,0.15)$);
\fill[gray!30] ($(A1)+(0,0.1)$) rectangle node[pos=0.1,above]{\color{gray} $C_1$} ($(AE)+(0,0.15)$);
\fill[gray!80] ($(A2)+(0,0.1)$) rectangle node[pos=0.08,above]{\color{gray} $C_2$} ($(AE)+(0,0.15)$);
\fill[gray!30] ($(A3)+(0,0.1)$) rectangle node[pos=0.08,above]{\color{gray} $C_3$} ($(AE)+(0,0.15)$);
\fill[gray!80] ($(AN)+(0,0.1)$) rectangle node[midway,above]{\color{gray} $\cap A_k$} ($(AE)+(0,0.15)$);
\fill[white] ($(A4)+(0,0.1)$) rectangle ($(AN)+(0,0.15)$);

\fill[gray!30] ($(B1)-(0,0.1)$) rectangle node[pos=0.1,below]{\color{gray} $C_1$} ($(BE)-(0,0.15)$);
\fill[gray!80] ($(B2)-(0,0.1)$) rectangle node[pos=0.08,below]{\color{gray} $C_2$} ($(BE)-(0,0.15)$);
\fill[gray!30] ($(B3)-(0,0.1)$) rectangle node[pos=0.08,below]{\color{gray} $C_3$} ($(BE)-(0,0.15)$);
\fill[gray!80] ($(BN)-(0,0.1)$) rectangle node[midway,below]{\color{gray} $\cap A_k$} ($(BE)-(0,0.15)$);
\fill[white] ($(B4)-(0,0.1)$) rectangle ($(BN)-(0,0.15)$);

\fill[gray!30] (4,0) -- (6,0) -- (6,-1.5) -- (4,-1.5);
\fill[gray!30] (2,-1.5) -- (3,-1.5) -- (3,0) -- (2,0);
\fill[gray!30] (3.5,-1.5) -- (3.75,-1.5) -- (3.75,0) -- (3.5,0);
\fill[gray!80] (0,0) -- (3,-1.5) -- (3.5,-1.5) -- (2,0);
\fill[gray!80] (3,0) -- (3.75,-1.5) -- (3.875,-1.5) -- (3.5,0);

\node at (-3,-1) {$h(x) = \left\{\begin{array}{ll} x, & x \in C_{2k+1} \vee x \in \cap A_k\\
                g(f(x)), & x \in C_{2k}\end{array}\right.$};
\end{tikzpicture}
\end{center}

Тогда определим $h(x)$ следующим образом:

$$h(x) = \left\{\begin{array}{ll} x, & x \in C_{2k+1} \vee x \in \cap A_k\\
                g(f(x)), & x \in C_{2k}\end{array}\right.$$

\hfill Q.E.D.

\subsection{Кардинальные числа}

\deff{def:} \textbf{Кардинальное число} --- наименьший ординал, не равномощный никакому меньшему:
$$\forall x.x \in c \rightarrow |x| < |c|$$

\thmm{Теорема.} Конечные ординалы --- кардинальные числа.

\deff{def:} Мощность множества $(|S|)$ --- равномощное ему кардинальное число.

\subsection{Диагональный метод}

\deff{def:} $|\mathbb{R}| > |\mathbb{N}|$

\textbf{Доказательство:}

Рассмотрим $a \in (0,1)$ и десятичную запись: $0.a_0a_1a_2\dots$.
Пусть существует биективная $f: \mathbb{N}\rightarrow (0,1)$.
По функции найдём значение $\sigma$, не являющееся образом никакого натурального числа.

\begin{center}
\begin{tabular}{cc|ccccccl}
 $n$ &  $f(n)$ & $f(n)_0$ & $f(n)_1$ & $f(n)_2$ & $f(n)_3$ & $f(n)_4$ & $f(n)_5$ & $\dots$ \\\hline
 $n_0$ &  0.3  &  \color{red}3    & 0    &  0  &  0  &  0  &  0 & $\dots$ \\
 $n_1$ & $\pi/10$ &  3  & \color{red}1    &  4  &  1  &  5  &  9 & $\dots$ \\
 $n_2$ & $1/7$   & 1 &   4    &  \color{red}2  &  8  &  5  &  7 & $\dots$ \\\hline
       & $\sigma$ & 8 &  6 &        7 &   \multicolumn{4}{l}{$\dots \sigma_k = (f(n_k)_k+5) \% 10$}
\end{tabular}
\end{center}

\hfill Q.E.D.

\thmm{Теорема Кантора}

$|\mathcal{P}(S)| > |S|$

\textbf{Доказательство:}

Пусть $S = \{a,b,c,\dots\}$

\begin{center}
\begin{tabular}{c|cccl}
$n$ & $a \in f(n)$ & $b \in f(n)$ & $c \in f(n)$ & $\dots$ \\\hline
$a$ & \color{red}И & Л & И \\
$b$ &       Л & \color{red}Л& И \\
$c$ &   И & И & \color{red}И\\\hline
   & Л & И & Л & $y \notin f(y)$
\end{tabular}
\end{center}

Пусть $f: S \rightarrow \mathcal{P}(S)$ --- биекция. Тогда $\sigma = \{ y\in S\ |\ y\notin f(y)\}$. Пусть $f(x) = \sigma$.
Но $x \in f(x)$ тогда и только тогда, когда $x \notin \sigma$, то есть $f(x) \ne \sigma$.

\hfill Q.E.D.

\subsection{Иерархии $\aleph_n$ и $\beth_n$}

\deff{def:} $\aleph_0 := |\omega|$; $\aleph_{k+1} := \min\{ a\ |\ a\text{ -- ординал},\aleph_k < |a|\}$

\deff{def:} $\beth_0 := |\omega|$; $\beth_{k+1} := |\mathcal{P}(\beth_k)|$

Континуум-гипотеза (Г.Кантор, 1877): $\aleph_1 = \beth_1$ (не существует мощности, промежуточной между счётной и континуумом).

Обобщённая континуум-гипотеза: $\aleph_n = \beth_n$ при всех $n$.

\deff{def:} Утверждение $\alpha$ противоречит аксиоматике: $\vdash\alpha$ ведёт к противоречию.

\deff{def:} Утверждение $\alpha$ не зависит от аксиоматики: $\not\vdash\alpha$ и $\not\vdash\neg\alpha$.

\thmm{Теорема о независимости континуум-гипотезы} 

Утверждение $\aleph_1 = \beth_1$ не зависит от аксиоматики ZFC.

\subsection{Примеры мощностей множеств}

\begin{center}
\begin{tabular}{l|l}
Пример & мощность\\\hline
$\omega$ & $\aleph_0$\\
$\omega^2$, $\omega^\omega$ & $\aleph_0$\\
$\mathbb{R}$ & $\beth_1$\\
все непрерывные функции $\mathbb{R}\rightarrow\mathbb{R}$ & $\beth_1$\\
все функции $\mathbb{R}\rightarrow\mathbb{R}$ & $\beth_2$
\end{tabular}
\end{center}

\subsection{Арифметика для кардинальных чисел}

\deff{def:} Если $\alpha$ и $\beta$ --- кардинальные числа, то 
$\alpha + \beta := |\alpha\uplus\beta|$,
$\alpha \cdot \beta := |\alpha \times \beta|$, 
$\alpha ^\beta$ --- мощность множества всех функций из $\beta$ в $\alpha$

\thmm{Теорема.} Если $\alpha$ --- некоторое бесконечное кардинальное число, то $\alpha\cdot\alpha = \alpha$

\thmm{Теорема.} Если $0 < \beta \le \alpha$ и $\alpha$ бесконечное, то $\alpha\cdot\beta = \alpha$

\textbf{Доказательство:}
\begin{itemize}
\item $\alpha \cdot \beta \ge \alpha$: фиксируем $b_0 \in \beta$ (т.к. $\beta > 0$), тогда в качестве $f : \alpha \rightarrow \alpha \times \beta$ возьмём $f(a) = \langle a,b_0\rangle$.
\item $\alpha \cdot \beta \le \alpha \cdot \alpha = \alpha$.
\end{itemize}
\hfill Q.E.D.

\subsection{Как пересчитать вещественные числа (неформально)?}

\begin{enumerate}
\item Номер вещественного числа --- первое упоминание в литературе, т.е. $\langle j, y, n, p, r, c \rangle$:\\
j --- гёделев номер названия научного журнала (книги);\\
y --- год издания;\\
n --- номер;\\
p --- страница;\\
r --- строка;\\
c --- позиция
\item Попробуете предъявить число $x$, не имеющее номера? Это рассуждение сразу даст номер.\\
\end{enumerate}

\subsection{Мощность модели и аксиоматизации}

\deff{def:} Пусть задана модель $\langle D, F_n, P_n \rangle$ для некоторой теории первого порядка. Её мощностью будем считать мощность $D$.


\deff{def:} Пусть задана формальная теория с аксиомами $\alpha_n$. Её мощность --- мощность множества $\{\alpha_n\}$.


Формальная арифметика, исчисление предикатов, исчисление высказываний --- счётно-аксиоматизируемые.


\subsection{Элементарная подмодель}

\deff{def:} $\mathcal{M}' = \langle D', F'_n, P'_n \rangle$ --- элементарная подмодель $\mathcal{M} = \langle D, F_n, P_n \rangle$, если:
\begin{enumerate}
\item $D' \subseteq D$, $F'_n$, $P'_n$ --- сужение $F_n$, $P_n$ (замкнутое на $D'$).
\item $\mathcal{M}\models \varphi(x_1,\dots,x_n)$ тогда и только тогда, когда $\mathcal{M}'\models \varphi(x_1,\dots,x_n)$ при $x_i \in D'$.
\end{enumerate}

\thmm{Теорема Лёвенгейма-Сколема}

Пусть $T$ --- множество всех формул теории первого порядка. Пусть теория имеет некоторую модель $\mathcal{M}$. Тогда найдётся элементарная подмодель $\mathcal{M'}$, причём $|\mathcal{M'}| \leq \max(\aleph_0, |T|)$.

\textbf{Доказательство(схема)}

\begin{enumerate} 
\item Построим $D_0$ --- множество всех значений, которые упомянуты в языке теории.
\item Будем последовательно пополнять $D_i$: $D_0 \subseteq D_1 \subseteq D_2 \dots$, следя за мощностью. $D' = \cup D_i$.
\item Покажем, что $\langle D', F_n, P_n\rangle$ --- требуемая подмодель.
\end{enumerate}


Пусть $\{f^0_k\}$ --- все 0-местные функциональные символы теории.
\begin{enumerate}
\item $D_0 = \{ \llbracket f^0_k \rrbracket \}$, если есть хотя бы один $f^0_k$.
\item Если таких $f^0_k$ нет, возьмём какое-нибудь одно значение из $D$.
\end{enumerate}

Очевидно, $|D_0| \le |T|$.

Фиксируем некоторый $D_k$. Напомним, $T$ --- множество всех формул теории. Рассмотрим $\varphi \in T$.
\begin{enumerate}
\item $\varphi$ не имеет свободных переменных --- пропустим.
\item $\varphi$ имеет хотя бы одну свободную переменную $y$.
\begin{enumerate}
\item $\varphi (y, x_1, \dots, x_n)$ при $y,x_i \in D_k$ бывает истинным и ложным --- ничего не меняем
\item $\varphi (y, x_1, \dots, x_n)$ при $y \in D$ и $x_i \in D_k$ либо всегда истинен, либо всегда ложен --- ничего не меняем
\item $\varphi (y, x_1, \dots, x_n)$ при $y,x_i \in D_k$ тождественно истинен или ложен, но при $y' \in D \setminus D_k$ отличается --- добавим $y'$ к $D_{k+1}$. Вместе добавим всевозможные $\llbracket\theta(y')\rrbracket$.
\end{enumerate}
\end{enumerate}

\begin{enumerate}
\item Всего добавили не больше $|T| \cdot |T|$ (для каждой формулы $\varphi$, возможно, будет добавлен $y$ --- и всевозможные выражения $\theta(y)$, допустимые в языке), и $|D_0| \le |T| \le |T|\cdot|T|$, отсюда $|D_k| \le |T| \cdot |T|$.
\item $|D'| = |\bigcup D_i| \le |T| \cdot |T| \cdot \aleph_0$.
\item Тогда $|T| \cdot |T| \cdot \aleph_0 = \max(|T|,\aleph_0)$. Разберём случаи:
\begin{enumerate}
\item Если $|T| < \aleph_0$, тогда $(|T| \cdot |T|) \cdot \aleph_0 = \aleph_0$
\item Если $|T| \ge \aleph_0$, тогда $(|T| \cdot |T|) \cdot \aleph_0 = |T| \cdot \aleph_0 = |T|$.
\end{enumerate}
\item Итого, $|D'| \le \max(|T|,\aleph_0)$.
\end{enumerate}

Докажем, что $\mathcal{M}'$ --- элементарная подмодель

Индукцией по структуре формул $\tau \in T$ покажем, что все формулы можно вычислить, и что $\llbracket \varphi \rrbracket_\mathcal{M'} = \llbracket \varphi \rrbracket_\mathcal{M}$.

\begin{enumerate}
\item База, 0 связок. $\tau \equiv P(f_1(x_1,\dots,x_n),\dots,f_n(x_1,\dots,x_n))$. Если $x_i \in D'$, то значит, добавлены на некоторых шагах (максимальный пусть $t$). Поэтому в $D_{t+1}$ можно вычислить формулу, и её значение сохранилось.
\item Переход. Пусть формулы из $k$ связок сохраняют значения. Рассмотрим $\tau$ с $k+1$ связкой.
\begin{enumerate}
\item $\tau \equiv \rho \star \sigma$ --- очевидно.
\item $\tau\equiv\forall y.\varphi(y,x_1,\dots,x_n)$. Каждый $x_i$ добавлен на каком-то шаге --- максимум $t$. Если $\varphi(y,x_1,\dots,x_n)$ бывает истинен и ложен при $y_t, y_f \in D$, то $y_t, y_f \in D_{t+1}$ (по построению). Поэтому, если $\mathcal{M}\not\models\forall y.\varphi(y,x_1,\dots,x_n)$, то и $\mathcal{M'}\not\models\forall y.\varphi(y,x_1,\dots,x_n)$. Если же $\varphi(y,x_1,\dots,x_n)$ не меняется от $y$, то тем более $\llbracket \varphi \rrbracket_\mathcal{M'} = \llbracket \varphi \rrbracket_\mathcal{M}$.
\item $\tau\equiv\exists y.\varphi(y,x_1,\dots,x_n)$ --- аналогично.
\end{enumerate}
\end{enumerate}

\subsection{<<Парадокс>> Сколема}

\begin{enumerate}
\item Как известно, $|\mathbb{R}| = |\mathcal{P}(\mathbb{N})| > |\mathbb{N}| = \aleph_0$. Однако, ZFC --- теория со счётным количеством формул. Значит, существует счётная модель ZFC, то есть $|\mathbb{R}| = \aleph_0$. В чём ошибка?
\item У равенств разный смысл, первое --- в предметном языке, второе --- в метаязыке. 
\end{enumerate}

